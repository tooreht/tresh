\section{Setup Raspberry Pi (RPI)}\label{setup-raspberry-pi-rpi}

\subsection*{Installation}\label{installation}

\begin{enumerate}
\def\labelenumi{\arabic{enumi}.}
\item
  Prepeare \texttt{cmdline.txt} of raspian image on sd card. Append the
  following:

\begin{verbatim}
ip=192.168.2.2
\end{verbatim}
\item
  (OSX) Internet Sharing WiFi to (Thunderbolt) Ethernet
\item
  Login to rpi (pw: raspberry)

\begin{verbatim}
ssh pi@192.168.2.2
\end{verbatim}
\item
  Set locales: Choose \texttt{en\_US.UTF8} in

\begin{verbatim}
sudo dpkg-reconfigure locales
\end{verbatim}
\item
  Set timezone: Choose \texttt{Zurich} in

\begin{verbatim}
dpkg-reconfigure tzdata
\end{verbatim}
\item
  Install required debian packages

\begin{verbatim}
aptitude install autossh git vim mosh htop # Not nodejs npm!!!
\end{verbatim}
\item
  Get latest Node.js debian package

\begin{verbatim}
wget http://node-arm.herokuapp.com/node_latest_armhf.deb
\end{verbatim}
\item
  Install Node.js

\begin{verbatim}
sudo dpkg -i node_latest_armhf.deb`
\end{verbatim}
\item
  Checkout source code

\begin{verbatim}
mkdir ~/src
cd src/
git clone <tresh_repo_url>
\end{verbatim}
\end{enumerate}

\subsection*{Configuration}\label{configuration}

\subsubsection*{DNS}\label{dns}

Add following line to network interface in /etc/network/interfaces:

\begin{verbatim}
dns-nameservers 8.8.8.8 8.8.4.4
\end{verbatim}

\subsubsection*{SSH Reverse Tunnel}\label{ssh-reverse-tunnel}

\begin{enumerate}
\def\labelenumi{\arabic{enumi}.}
\item
  Add autossh-tresh systemd service

\begin{verbatim}
pi@raspberrypi:~ $ vim /etc/systemd/system/autossh-tresh.service

[Unit]
Description=Autossh tresh: Maintain a ssh reverse tunnel to a control server
# After=network.target
After=network-online.target

[Service]
# User=autossh
User=pi
Group=pi
# -p [PORT]
# -l [user]
# -M 0 --> no monitoring
# -N Just open the connection and do nothing (not interactive)
# LOCALPORT:IP_ON_EXAMPLE_COM:PORT_ON_EXAMPLE_COM
# ExecStart=/usr/bin/autossh -M 0 -N -q -o "ServerAliveInterval 60" -o "ServerAliveCountMax 3" -p 22 -l autossh remote.example.com -L 7474:127.0.0.1:7474 -i /home/autossh/.ssh/id_rsa
TimeoutStartSec=3
TimeoutStopSec=3
ExecStart=/usr/bin/autossh -M 20000 -N -C -q -o "ServerAliveInterval 60" -o "ServerAliveCountMax 3" -p 22 -l tooreht tjmp -R 22000:localhost:22 -i /home/pi/.ssh/id_rsa
# ExecStart=/home/pi/scripts/autossh-tresh.sh
Restart=always

[Install]
WantedBy=multi-user.target
\end{verbatim}
\item
  Enable service

\begin{verbatim}
systemctl enable autossh-tresh.service
\end{verbatim}
\end{enumerate}

\subsubsection*{Detect Waspmote Gateaway USB via
udev}\label{detect-waspmote-gateaway-usb-via-udev}

\begin{enumerate}
\def\labelenumi{\arabic{enumi}.}
\item
  Add udev rule

\begin{verbatim}
pi@raspberrypi:~ $ vim  /etc/udev/rules.d/99-tresh.rules

SUBSYSTEM=="tty", ACTION=="add", ATTRS{manufacturer}=="FTDI", ATTRS{product}=="FT232R USB UART",\
  PROGRAM="/bin/systemd-escape -p --template=tresh-gateway@.service $env{DEVNAME} $env{ID_FS_LABEL}",\
  ENV{SYSTEMD_WANTS}+="%c"
\end{verbatim}
\item
  Add tresh-gateway@ systemd service

\begin{verbatim}
pi@raspberrypi:~ $ vim /etc/systemd/system/tresh-gateway@.service

[Unit]
Description=tresh siot gateway
BindTo=%i.device
After=%i.device
After=rc-local.service

[Service]
Type=oneshot
TimeoutStartSec=0
ExecStart=/home/pi/scripts/tresh-gateway.sh
\end{verbatim}
\end{enumerate}
