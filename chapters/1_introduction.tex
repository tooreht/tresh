\chapter{Einleitung}

\acrfull{iot} hat laut Gartner\autocite{gartner:iotHype} gerade den Höhepunkt des Hypes erreicht. Der Begriff ist in aller Munde und jeder kennt es. Doch wo sind solche "Things" tatsächlich bereits anzutreffen? Diese Arbeit behandelt das Thema "Metropolitan-Sensor/Aktor-Netze unter Einsatz von \gls{lora} und anderen Techniken". Sie sucht nach Anwendungsfällen wo solche Things, auch \gls{iotk} genannt, im urbanen Umfeld sinnvoll eingesetzt werden können.\\
Ein \gls{iotk} ist ein Mikrocontroller, welcher Sensoren und Aktoren verwaltet. Die Sensordaten sendet er über eine Verbindung zu einem Gateway, welches die Daten wiederum an einen Server oder in die Cloud leitet. Umgekehrt können vom Server oder von der Cloud Befehle über das Gateway an den Controller geschickt werden. Der Controller steuert damit die Aktoren. Aus solchen \gls{iotk} und  Gateways entsteht schliesslich ein Sensor-Aktor-Netz. Für dieses Netz spielt die Verbindung zwischen den IoT-Knoten und den Gateways eine sehr wichtige Rolle. Sie definiert Faktoren wie Distanz, Bandbreite und Energieverbrauch. Diese Faktoren sind für einen \gls{iotk} von entscheidender Bedeutung da sie direkten Einfluss auf dessen Lebensdauer und seine Kosten haben. Ebenso beschränkt es die Grösse des Netzes, je nachdem welche Reichweite mit der Verbindung möglich ist. Da die \gls{iotk} oft an Orten eingerichtet werden wo weder eine Netzwerk-Anbindung noch Energieversorgung verfügbar ist, muss der Sensor möglichst autark sein. Das bedingt eine drahtlose Verbindung zwischen der Knoten und der Gateways. Aus diesem Grund behandelt die Arbeit unter anderem verschiedene Funk-Standards, welche zurzeit existieren und in den Frequenzen eines \gls{ism}es sind, damit keine Lizenzen die Kosten der \gls{iotk} in die Höhe treiben. LoRa, eine Funkmodulationstechnik hat seine Stärken genau in den Faktoren Distanz und Energieverbrauch, weshalb sie ideal für solche Verbindungen geeignet scheint. Aus diesem Grund implementiert diese Arbeit ein Proof-of-concept mit diesem Standard. 