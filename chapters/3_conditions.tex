\chapter{Rahmenbedingungen für \gls{iot}}
Die Use Cases aus den verschiedensten Branchen zeigen: in fast jeder Lebenssituation können Dinge automatisiert oder verbessert werden können. So manchen Prozess, welchen wir heutzutage regelmässig wiederholen, kann automatisiert oder zumindest zeitlich optimiert werden, so dass er nur ausgeführt wird, wenn er wirklich nötig ist. Viele der beschriebenen Use Cases sind von mehreren Faktoren Abhängig. Der in Smart Home beschriebene Wecker benötigt beispielsweise folgende Sensoren und Aktoren:
\begin{itemize}  
  \item Zeit-Sensor (Real-Time Clock)
  \item Wetter-Sensor (Licht-Sensor vor dem Fenster)
  \item Schlafphasen-Sensor (Smartwatch)
  \item Aufsteh-Sensor (Bodenplatten-Sensoren oder Smartwatch)
  \item Storen-Aktor (Elektrisch gesteuerte Storen)
  \item Licht-Aktor (Smarte LEDs)
  \item Musik-Aktor (Smarte Stereoanlage)
\end{itemize}
Dieser Wecker funktioniert erst, wenn der Wecker alle diese Sensoren auslesen kann und die Aktoren steuern kann. Natürlich könnte ein Wecker konstruiert werden, welcher all diese Informationen für sich erfasst und so autonom funktioniert. Es existieren mittlerweile auch bereits alle Sensoren und Aktoren für sich. Allerdings wäre ein solch intelligenter Wecker als Komplettprodukt viel zu teuer und nicht erweiterbar. Deshalb ist es viel sinnvoller, wenn Sensoren ihre Daten an eine \gls{iot}-Platform senden, welche die Sensordaten über standardisierte Schnittstellen empfängt und danach für Aktoren wieder über Schnittstellen bereitstellt. Erst mit einer solchen Plattform kann \gls{iot} sein Potential ausschöpfen. Durch das Teilen von Informationen können Sensoren und Aktoren voneinander profitieren.
Somit muss nicht jedes System seine eigenen Sensoren und Aktoren zur Verfügung stellen. Redundanzen können verhindert werden. Dies ermöglicht, dass komplexe Systeme rasch und effizient entstehen können.  

Heutzutage existieren bereits viele \gls{iotk}. Beispielsweise kann jedes Smartphone als \gls{iotk} betrachtet werden. Smartphones haben beispielsweise Sensoren für Helligkeit, Temperatur, Beschleunigung Neigung, Position, Feuchtigkeit und Näherung. Da Smartphones einem grossen Teil der Zeit in Hand- oder Hosentaschen aufbewahrt werden, messen die Sensoren für die meisten Anwedungsfälle 
Die Daten dieser Knoten sind jedoch zum grössten Teil der Zeit   

Weiter sind die bereits vorhandenen Sensoren meist in einem in sich geschlossenen System, welches keine Schnittstellen anbietet. \gls{iot} kann seine  jedoch erst 
So ein System entwickelt das Unternehmen Appmodule. SIOT.net ist eine Plattform, 