\chapter{Drahtlose Datenübertragung}

Der Bedarf an drahtlosen Technologien im industriellen sowie im privaten Umfeld steigt immer mehr. Die Schlüsselfaktoren, die hinter den Anforderungen für drahtlose Technologien stehen, sind zum einen die Fernsteuerung, damit verbunden die Mobilität und Flexibilität, zum anderen die verschleissfreie Übertragung von Daten. Beispielsweise können drahtlose Sensoren und Aktoren, die auf beweglichen Teilen von Maschinen positioniert werden, von Vorteil für drahtlose Systeme sein. Jede Anwendung hat unterschiedliche Anforderungen. Es gibt kein einzelnes drahtloses System, das alle Anforderungen gleichzeitig erfüllen kann.\\
Im Rahmen dieser Arbeit wurden mehrere Funktechnologien die im \acrshort{iot}-Umfeld in Frage kommen angeschaut und deren Vor- und Nachteile aufgezeigt, die nachfolgend beschrieben werden.

\section{\acrshort{ble}}

Ursprünglich wurde \acrfull{ble} als \glqq{}Bluetooth 4.0\grqq{} bezeichnet, als es im Jahre 2011 auf den Markt kam. Doch die Ausrichtung auf geringen Energieverbrauch, was der Hauptvorteil gegenüber dem regulären Bluetooth Standard ist, war marketingtechnisch so wichtig, dass sich dies auch im Namen manifestieren musste. Weil eine einzelne Batterie ein Produkt das über \acrshort{ble} verbunden ist bis zu fünf Jahren betreiben kann, ist diese Technologie ideal für \acrshort{m2m}-Kommunikation.

\acrshort{ble} operiert im 2.4 GHz \gls{ism}, wie Bluetooth 1.0. Doch um Energie zu sparen, wurde die Übertragungszeit auf wenige Millisekunden, dafür aber mit einer hohen Datenrate von 1 Mbs optimiert. Danach geht \acrshort{ble} sofort in einen Schlafmodus bis wieder eine Verbindung aufgebaut wird.
Durch die begrenzte Reichweite von etwa 100m eignet sich \acrshort{ble} vor allem für Anwendungen im nahen Umkreis wie kontaktloses Zahlen, Dienste im öffentlichen Verkehr oder Aufzeichnen des Gesundheitszustands und Fitness.

\section{\gls{zigbee}}

\gls{zigbee} ist ein Mesh-Netzwerk Protokoll. Es ist dafür ausgelegt kleine Datenpakete über kurze Distanzen zu Übertragen und dabei möglichst wenig Energie zu verbrauchen. In einer Mesh-Topologie werden Daten eines einzelnen Sensors durch ein Netz von \gls{iotk} geleitet. Jeder \gls{iotk} fungiert dabei gleichzeitig als Datenquelle und Übermittler (Repeater) bis die Daten den \gls{iotg} erreicht haben.

\gls{zigbee} verwendet eine Version des \gls{ieee802.15.4} Standards und ist deshalb in lokalen Sensor Netzwerken, vor allem in der Gebäude Automation weit verbreitet. \gls{zigbee} verwendet das 2.4 GHz \gls{ism} und kann, weil dies ein internationaler Standard ist, praktisch überall auf der Welt eingesetzt werden.

\section{\gls{3gpp}}

\gls{3gpp} - Die Industrie Experten hinter der Standardisierung von Zellulären Netzwerken - haben kürzlich drei neue Standards vorgestellt:

\begin{enumerate}
  \item LTE-MTC (auch bekannt unter LTE-M)
  \item NB-LTE-M
  \item NB-IoT
\end{enumerate}

Dies sind die Bestrebungen der Telekommunikationsbranche Geräte die in einem Träger Netzwerk operieren günstiger und energieeffizienter zu machen, nicht zuletzt um Einfluss auf das Internet der Dinge (LTE \acrshort{iot}) und die \acrfull{m2m} Kommunikation zu gewinnen. Diese neuen Standards sind eine Antwort auf \gls{lora}, \gls{lorawan} und \gls{sigfox}, also die \gls{lpwan}-Technologien.

\subsection{LTE-MTC (LTE-M)}

LTE-M, eine abgekürzte Version von LTE-MTC (Machine-Type Communications) ist Teil von \gls{3gpp}'s Release 12 und 13 und ist immer noch in der Evaluationsphase. Der \gls{lte} Kanal besteht aus Ressource-Blöcken die je etwa 230 kHz des Frequenzspektrums besetzten. LTE-M ist Teil des 1.4 MHz Blocks, der aus sechs Ressource-Blöcken besteht. LTE-M ist aufgrund seines erweiterten, diskontinuierlichen Repetitionszykluses (DRX) energieeffizienter. Dies bedeutet, dass ein Endgerät mit der Basis Station vereinbaren kann, wie oft es aufwacht um Daten zu Empfangen.\\
Der Vorteil von LTE-MTC für \acrshort{m2m} Kommunikation ist, dass es sich in das bestehende Konstrukt von \gls{lte} Netzwerken eingliedert. In anderen Worten, der Netzwerk Provider muss nur seine Basisband-Software auf seinen Basis Stationen aktualisieren und muss keine Geld für neue Antennen ausgeben. Die Empfänger sind auch viel einfacher aufgebaut als z.B. eine Smartphone, denn sie müssen nur 1.4 MHz digitalisieren können, anstatt 20 MHz.\\
LTE-M hat eine leicht höhere Datenrate als NB-LTE und NB-IoT, es können relativ grosse Datenmengen übertragen werden. Darum ist LTE-M für Anwendungen wie Object Tracking, Wearables, Energiemanagement und Stadt-Infrastruktur geeignet. 

\subsection{NB-LTE-M}

Wie LTE-M ist NB-LTE-M ein anderer Overlay der existierenden LTE Struktur. Dabei werden bestehende Ressource-Blöcke für \gls{iot} Datenverkehr reserviert und kleinere Kanäle verwendet um einfachere Empfänger bauen zu können. NB-LTE-M schlägt die Wiederverwendung von GSM vor, indem ein einzelner ~200 kHz Ressource-Block verwendet wird.\\
Was NB-LTE-M einzigartig macht, ist dass anstatt das 1.4 MHz Spektrum und sechs Ressource-Blöcke zu nutzen, nur ein LTE Ressource-Blöcke verwendet wird. Dies ergibt für den Endbenutzer einen effektiven Durchsatz von etwa 200 kbps down und 144 kpbs up.

\subsection{NB-IoT}

NB-IoT ist ein anderer \gls{3gpp} Release 13 Vorschlag der nicht auf LTE, sondern auf einer DSSS (Direct Sequence Spread Spectrum) Modulation basiert. Die Komplexität um Endgeräte dafür zu bauen ist noch geringer als bei NB-LTE-M und zusätlich können auch kostengünstigere Chipsets verwendet werden. Befürworter von NB-IoT, wie Huawei, Ericsson, Qualcomm, und Vodafone, arbeiten aktiv zusammen um diesen Standard durchzusetzen.\\
Die Schwierigkeit von NB-IoT ist die Verbreitung der Infrastruktur. Da es nicht Teil des LTE Standards ist, muss es entweder in einem Nebenband operieren, wozu andere Software benötigt wird, welche die Netzwerkanbieter viel Geld kosten kann oder es muss das veraltete GSM Spektrum wiederverwendet werden.

\section{\gls{wlan}}

\gls{wlan} verwendet Radiowellen um drahtlos zwischen zwei Geräten zu kommunizieren. Diese Technologie wird hauptsächlich verwendet um Computer, Laptops, Tablets, Smartphones, etc. über einen Internet Gatway, meist in Form eines Access Points, mit dem Internet zu verbinden. Grundsätzlich können aber beliebige zwei Hardware Komponenten miteinander verbunden werden. \gls{wlan} ist ein Drahtloses lokales Netzwerk das im \gls{ieee802.11} Standard spezifiziert ist.\\
\gls{wlan} unterstützt zwei \glspl{ism}; das globale 2.4 GHz UHF ISM und das 5 GHz SHF Radioband. Die WiFi Allianz zertifiziert einige Produkte, die dann mit dem \glqq{}Wi-Fi Certified\grqq{} bezeichnet werden dürfen. Um dieses Zertifikat zu erhalten, muss das Produkt den Interoperabilitäts-Tests der Allianz genügen.

\section{\gls{sigfox}}

\gls{sigfox} ist eine Ultra-Schmalband Technologie die von der gleichnamigen Firma entwickelt wird. Das Unternehmen baut eine eigene Netzinfrastruktur mit Partnern auf und vermietet diese dann als Service inklusive Cloud Plattform. Wie bei Telekom-Anbietern wird die Abdeckung durch Funkantennen gewährleistet und ist bereits in 20 Ländern verfügbar.\\
\gls{sigfox} verwendet unlizenzierte \glspl{ism}; 868 MHz in Europa und 902 MHz in den USA.
\gls{sigfox} nutzt die Standard Modulationsverfahren BPSK (Binary Phase-Shift Keying) für den Uplink und GFSK (Gaussian Frequency-Shift Keying) für den Downlink. Dabei verwendet \gls{sigfox} ganz kleine Teile des Frequenzspektrums und ändert die Phase der Trägerradiowelle um die Daten zu encodieren. Dies erlaubt dem Empfänger nur auf einem ganz kleinen Abschnitt des Spektrums zu hören, was den Effekt des Rauschens abschwächt.
Dies erklärt die Netzwerk Performance von \gls{sigfox}:

\begin{itemize}
  \item Bis zu 140 Nachrichten pro \gls{iotk} pro Tag
  \item Payload Grösse von 12 Byte für jede Nachricht
  \item Datendurchsatz von bis zu 100 bit pro Sekunde
\end{itemize}

Mit einer Sensitivität von 162 dB kann über sehr weite Strecken gefunkt werden. In ländlichen Gebieten soll die Reichweite 30-50 km betragen, in urbaneren Gebieten durch diverse Hindernisse und höherem Rauschen nur noch zwischen 3 und 10 km. Auf offenem Gelände mit Sichtkontakt sollen laut \gls{sigfox} sogar Reichweiten von über 1000 km erreicht werden.\\
Während die Endpunkt-Geräte sehr simpel und dadurch auch kostengünstig sind, werden dafür besser ausgestattete Basis Stationen benötigt. Die \gls{sigfox} Kommunikation funktioniert tendenziell besser von den Endpunkten zur Basistation. Bidirektionale Kommunikation wird zwar unterstützt, aber die Kapazität von der Basis Station zurück zum Endpunkt ist beschränkt und auch das Downlink Budget ist geringer. Dies ist auf die geringere Empfangssensitivität des Endpunkts gegenüber der teuren Basis Station zurückzuführen.

\section{\gls{lora}}

\gls{lora} ist der physikalische Layer oder die drahtlose Modulationstechnologie um den Kommunikationslink über weite Strecken aufzubauen. Viele veraltete Drahtlossysteme nutzen das FSK (Frequency-Shift Keying) Modulationsverfahren, weil es sehr energieeffizient ist. \gls{lora} basiert auf dem CSS (Chirp Spread Spectrum) Modulationsverfahren welches die gleichen energieeffizienten Charakteristiken aufweist wie FSK Moudulation, aber die Reichweite massiv erweitert. CSS ist beim Militär und zur Kommunikation im Weltraum schon seit Jahrzehnten im Einsatz, gerade wegen den langen Kommunikationsdistanzen die erreicht werden können und der Robustheit gegenüber Interferenz. \gls{lora} ist die erste kostengünstige Implementation für den kommerziellen Einsatz.    

Der Vorteil von \gls{lora} ist die auf weite Distanzen ausgelegte Technologie. Ein einzelnes Gateway oder Basis Station kann ganze Städte oder hunderte Quadratkilometer abdecken. Die Reichweite ist stark von der Umgebung und der sich in ihr befindenden Hindernissen abhängig. Aber \gls{lora} besitzt eine sehr hohe Sensitivität, je nach Modul bis zu -168 dBm Link Budget\footnote{Quelle: \url{http://www.slideshare.net/zahidtg/lora-introduction} Slide 11}. 

\section{\gls{lorawan}}

\gls{lorawan} baut auf \gls{lora} auf und definiert das Kommunikationsprotokoll und die Systemarchitektur des Netzwerks. Das Protokoll und die Systemarchitektur haben den grössten Einfluss auf die Batterielaufzeit eines \gls{iotk}, die Netzwerkkapazität die Servicequalität, die Sicherheit und die Vielfältigkeit der Anwendungen die das Netzwerk unterstützt.

\begin{figure}[H]
     \centering
        \includegraphics[width=1.0\textwidth]{pictures/lorawan-architecture.jpg}
    \caption{LoRaWAN Architektur}
    \label{fig:LoRaWAN Architektur}
\end{figure}

Viele bestehende Netzwerke verwenden eine Mesh-Netzwerk Architektur. In einem Mesh-Netzwerk leiten die einzelnen \gls{iotk} die Information an andere Knoten weiter um die Reichweite und Zellengrösse des Netzwerks zu vergrössern. Während dies die Reichweite verbessert, nimmt auch die Komplexität zu, reduziert die Netzwerkkapazität und verkürzt die Batterielaufzeit, da die Knoten Daten von anderen Knoten empfangen und weiterleiten, die für sie selber irrelevant ist. Eine auf weite Distanzen ausgelegte Stern-Architektur macht am meisten Sinn um die Batterielaufzeit hoch zu halten wenn dafür Verbindungen über lange Distanzen erreicht werden können. Die Kommunikation der Knoten in \gls{lorawan} ist asynchron, sie senden nur Daten wenn diese verfügbar sind, sei es ereignisbasiert oder in gewissen zeitlichen Abständen. Diese Art von Protokoll wird typischerweise als Aloha Methode bezeichnet.

\subsubsection*{Netzwerk Architektur}

In einem \gls{lorawan} Netzwerk sind die \gls{iotk} nicht einem spezifischen Gateway zugeordnet. Stattdessen werden die von einem \gls{iotk} übermittelten Daten von mehreren Gateways empfangen. Jedes Gateway sendet die von einem Endknoten empfangenen Daten weiter zu einem Cloud basierten Netzwerk Server über einen Backhaul wie Zelluläres Netzwerk, Ethernet, Satellit oder \gls{wlan}. Die Intelligenz und Komplexität wird auf den Netzwerk Server verschoben. Er übernimmt das Deduplizieren von redundanten Packeten, überprüft die Sicherheit, legt die Empfangsbestätigung über den optimalen Gateway fest und weist den \gls{iotk} adaptiv die beste Datenrate zu, etc. Wenn ein Knoten mobil ist oder sich ständig bewegt ist kein Handover von Gateway zu Gateway nötig, was ein kritische Anforderung ist, um Güter zu überwachen, ein bedeutender Anwendungsfall für \acrshort{iot}.

\begin{figure}[H]
     \centering
        \includegraphics[width=1.0\textwidth]{pictures/lorawan-network-architecture.png}
    \caption{LoRaWAN Netzwerk Architektur}
    \label{fig:LoRaWAN Netzwerk Architektur}
\end{figure}

\subsubsection*{Netzwerkkapazität}

Um ein Netzwerk das auf lange Distanzen ausgelegt ist zu realisieren, muss der Gateway eine sehr hohe Kapazität haben oder die Fähigkeit haben Pakete von sehr vielen Knoten zu empfangen. Eine hohe Netzwerkkapazität wird in \gls{lorawan} durch die Verwendung von adaptiven Datenraten und mit Multikanal- und Multimodem Transceiver, so dass gleichzeitig Pakete auf mehreren Kanälen empfangen werden können. Die kritischen Faktoren die die Kapazität betreffen sind die Anzahl simultaner Kanäle, Datenrate (Zeit in der Luft), die Payload-Länge und wie häufig die Knoten Daten versenden. Da \gls{lora} eine Frequenzspreizungsmodulation ist, stehen die Signale praktisch orthogonal zueinander wenn unterschiedliche Spreizfaktoren verwendet werden. Wenn der Spreizfaktor ändert, ändert auch die effektive Datenrate. Ein Gateway nutzt diese Eigenschafte zu seinem Vorteil indem er mehrere unterschiedliche Datenraten auf dem gleichen Kanal zur gleichen Zeit empfangen kann. Wenn der Knoten eine gute Verbindung hat und in der Nähe eines Gateways steht, gibt es keinen Grund warum er immer die tiefste Datenrate verwendet sollte, die nur das verfügbare Spektrum länger als nötig besetzt. Beim erhöhen der Datenrate wird die Zeit in der das Paket in der Luft ist kürzer, während mehr potentieller Platz für Pakete anderer Knoten übrig bleibt. Die adaptive Datenrate verlängert auch die Batterielaufzeit eines Knotens. Um eine adaptive Datenrate zu ermöglichen wird ein symmetrischer Up- und Downlink mit genügend genügend Downlink Kapazität benötigt. Diese Merkmale ermöglichen einem \gls{lorawan} Netzwerk eine sehr hohe Kapazität zu haben und skalierbar zu machen. Ein Netzwerk kann mit minimaler Infrastruktur einfach aufgebaut werden und wenn mehr Kapazität benötigt wird können einfach mehr Gateways hinzugefügt werden. Dies erhöt die Datenraten, reduziert das Übersprechen auf andere Gateways und die Kapazität skaliert 6-8 mal. Alternative \gls{lpwan} Technologien bieten nicht die Skalierbarkeit von \gls{lorawan}, meist wegen technologischen Kompromissen, welche die Downlink Kapazität einschränken oder das Verhältnis zwischen Up- und Downlink asymmetrisch machen.

\subsubsection*{Geräteklassen}

Endgeräte müssen verschiedene Aufgaben erfüllen und haben unterschiedliche Anforderungen. Um dieser Vielfalt von Anwendungsprofilen gerecht zu werden sind in \gls{lorawan} drei verschiedene Geräteklassen definiert. Diese variieren zwischen schneller Reaktionszeit vom Server zum \gls{iotk} und weniger Energieverbrauch der Endgeräte. In einer regel- oder aktorbasierten Anwendung ist die Reaktionszeit im Netzwerk ein zentraler Faktor. Für \gls{iotk} die nur sporadisch Sensordaten an den Server übermitteln ist die Batterielaufzeit viel wichtiger.

\textbf{Klasse A:} Endgeräte der Klasse A erlauben eine bidirektionale Kommunikation. Auf die Uplink Übertragung jedes Endgeräts folgen zwei kurze Downlink Empfangsfenster. Der Übertragungszeitpunkt wird je nach seinen Bedürfnissen vom Endgerät, mit einer kleinen Variation basierend auf einer zufälligen Wartezeit (ALOHA-Typ Protokoll), gesteuert. Die Klasse A Operation ist die energieeffizienteste für Endgeräte, welche nur Downlink Kommunikation mit dem Server, kurz nach jedem Übermitteln der Daten benötigen. Die Downlink Kommunikation vom Server muss zu jeder anderen Zeit bis zur nächsten geplanten Uplink Übertragung warten.

\textbf{Klasse B:} Endgeräte mit bidirektionaler Kommunikation und festgesetzten Empfangszeitfenstern. Zusätzlich zu zufälligen Empfangsfenstern der Klasse A, öffnen Klasse B Endgeräte weitere Empfangszeiten zu festgelegten Zeitpunkten. Damit Endgeräte wissen wann sie ihr Empfangsfenster zur festgelegten Zeit starten sollen, erhalten sie ein Zeit synchronisiertes Beacon vom Gateway. Dies erlaubt dem Server herauszufinden ob sich das Endgerät im Empfangszustand befindet.

\textbf{Klasse C:} Endgeräte mit bidirektionaler Kommunikation und maximalen Empfangsfenstern. Endgeräte der Klasse C haben fast kontinuierliche offene Zeitfenster, sie sind nur geschlossen, wenn das Endgerät Daten überträgt.

\begin{figure}[H]
     \centering
        \includegraphics[width=1.0\textwidth]{pictures/device-classes.png}
    \caption{LoRaWAN Geräteklassen}
    \label{fig:LoRaWAN Geräteklassen}
\end{figure}

\begin{landscape}

\begin{table}[H]
\centering
% A table with adjusted row and column spacings
% \setlength sets the horizontal (column) spacing
% \arraystretch sets the vertical (row) spacing
\begingroup
\setlength{\tabcolsep}{10pt} % Default value: 6pt
\renewcommand{\arraystretch}{1.5} % Default value: 1
\begin{tabulary}{\paperwidth}{m{3cm} m{2cm} m{4cm} m{4cm} C m{2cm}}

\textbf{Modell}            & \textbf{Protokoll}                & \textbf{Frequenz}                                                                                        & \textbf{txPower}                                                               & \textbf{Sensivität} & \textbf{Reichweite}               \\ \hline
XBee-802.15.4-Pro          & 802.15.4                          & 2.4GHz                                                                                                   & 100mW                                                                          & -100dBm             & 7000m                             \\
XBee-ZB-Pro                & ZigBee-Pro                        & 2.4GHz                                                                                                   & 50mW                                                                           & -102dBm             & 7000m                             \\
XBee-868                   & RF                                & 868MHz                                                                                                   & 315mW                                                                          & -112dBm             & 12km                              \\
XBee-900                   & RF                                & 900MHz                                                                                                   & 50mW                                                                           & -100dBm             & 10Km                              \\
LoRaWAN                    & LoRaWAN                           & 868 and 433MHz. 900- 915MHz version coming in 2016.                                                      & 14dBm                                                                          & -136dBm             & - km - Typical base station range \\
LoRa                       & RF                                & 868 and 915 MHz                                                                                          & 14 dBm                                                                         & -137dBm             & 21+Km                             \\
Sigfox                     & Sigfox                            & 868MHz                                                                                                   & 14 dBm                                                                         & -126dBm             & - km - Typical base station range \\
WiFi                       & 802.11b/g                         & 2.4GHz                                                                                                   & 0dBm - 12dBm                                                                   & -83dBm              & 50m-500m                          \\
GPRS Pro and GPRS+GPS      & -                                 & 850MHz/900MHz/ 1800MHz/1900MHz                                                                           & 2W(Class4) 850MHz/900MHz, 1W(Class1) 1800MHz/1900MHz                           & -109dBm             & - Km - Typical carrier range      \\
3G/GPRS                    & -                                 & Tri-Band UMTS 2100/1900/900MHz Quad-Band GSM/EDGE, 850/900/1800/1900 MHz                                 & UMTS 900/1900/2100 0,25W GSM 850MHz/900MHz 2W DCS1800MHz/PCS1900MHz 1W         & -106dBm             & - Km - Typical carrier range      \\
Bluetooth Low Energy       & Bluetooth v.4.0 / Bluetooth Smart & 2.4GHz                                                                                                   & 3dBm                                                                           & -103dBm             & 100m
\end{tabulary}
\endgroup
% The \begingroup ... \endgroup pair ensures the separation
% parameters only affect this particular table, and not any
% sebsequent ones in the document.
\caption{IoT RF-Technologien}
\label{fig:IoT RF-Technologien}
\end{table}

\end{landscape}