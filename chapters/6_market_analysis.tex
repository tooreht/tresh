\chapter{Der Use Case Tresh}
\section{Entwicklung des Use Case}
Durch den ersten Teil der Arbeit, der Analyse von Rahmenbedingungen, Anforderungen und technischen Möglichkeiten für Funktechnologien von \gls{iotk} hat sich gezeigt, dass \gls{lora} in praktisch allen Bereichen eine optimales Verhältnis zu allen Anforderungen hat. Nun ist als zweiter Teil der Arbeit das Ziel, anhand eines praktischen Use Cases diese Technologie effektiv einzusetzen und dadurch erste Erfahrungen damit zu sammeln. Mit der \gls{pshmthd} soll nun ein Bedürfnis gefunden werden, welches zu der \gls{lora}-Technologie passt. Dabei kommen als erstes natürlich die Use Cases aus dem Kapitel \ref{Anwendungsfälle für IoT} zur Evaluation. Es stellte sich aber heraus, dass die meisten Use Cases den Rahmen des Projekt 2 sprengen würden. Das grösste Problem vieler Use Cases sind die verschiedenen Abhängigkeiten. So wie in den Rahmenbedingungen in Kapitel \ref{Rahmenbedingungen für iot} beschrieben, hat ein intelligentes \gls{iot}-System schnell einmal Abhängigkeiten von mehreren Sensoren und Aktoren. Dies wiederum löst eine Kettenreaktion von zu verbindenden Schnittstellen aus, was    

\section{Markt-Recherchen}

\section{Bedürnis-Deckungs-Analyse beim Tiefbauamt}