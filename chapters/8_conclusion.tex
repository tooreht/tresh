\chapter{Fazit}

Wir haben im Rahmen dieser Projekt 2 Arbeit viele Erfahrungen zum Thema \gls{iot} gesammelt. Sei es die Auseinandersetzung mit möglichen Use-Cases, der Definition von Rahmenbedingungen für \gls{iot}, dem grundlegenden Aufbau einer \gls{iot}-Architektur, dem Herausfiltern der wichtigsten Anforderungen an \glspl{iotk}, ein Überblick über die verfügbaren \gls{iot} Funktechnologien und deren Vor- und Nachteile und schliesslich der Implementation und Evaluation einer eigenen kleinen \gls{iot}-Infrastruktur mit \gls{lora} anhand des Tresh Use-Case. Wir haben \gls{iot} also aus verschiedenen Blickwinkeln betrachtet und ein Gefühl für das Potential, die Herausforderungen und die Limitationen von \gls{iot} und \acrshort{m2m} Kommunikation entwickelt.\\
Im Nachhinein bereuen wir ein wenig, dass wir erst gegen Ende der Arbeit festgestellt haben, dass wir \gls{lorawan} zu wenig vertieft angeschaut haben. Stattdessen wollten wir die beiden verschieden Ansätze von \gls{lora} und \gls{zigbee} im Praxistest untersuchen. Aus zeitlichen Gründen haben wir dann ausschliesslich mit \gls{lora} gearbeitet. Die Fokussierung auf eine Technologie hätte uns die Arbeit erleichtert.\\
Trotzdem sind wir zufrieden mit unser Leistung, nicht zuletzt weil wir unseren Fehler erkannt und einiges daraus gelernt haben. Wir denken, dass wir mit dieser Arbeit eine gute Grundlage geschaffen haben, um die ganze Thematik im Rahmen unserer Bachelor-Thesis zu vertiefen.
