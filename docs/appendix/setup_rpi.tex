\section{Setup Raspberry Pi (RPI)}\label{setup-raspberry-pi-rpi}

\subsection*{Installation}\label{installation}

\begin{enumerate}
\def\labelenumi{\arabic{enumi}.}
\item
  Prepeare \texttt{cmdline.txt} of raspian image on sd card. Append the
  following:

\begin{verbatim}
ip=192.168.2.2
\end{verbatim}
\item
  (OSX) Internet Sharing WiFi to (Thunderbolt) Ethernet
\item
  Login to rpi (pw: raspberry)

\begin{verbatim}
ssh pi@192.168.2.2
\end{verbatim}
\item
  Set locales: Choose \texttt{en\_US.UTF8} in

\begin{verbatim}
sudo dpkg-reconfigure locales
\end{verbatim}
\item
  Set timezone: Choose \texttt{Zurich} in

\begin{verbatim}
dpkg-reconfigure tzdata
\end{verbatim}
\item
  Install required debian packages

\begin{verbatim}
aptitude install autossh git vim mosh htop # Not nodejs npm!!!
\end{verbatim}
\item
  Get latest Node.js debian package

\begin{verbatim}
wget http://node-arm.herokuapp.com/node_latest_armhf.deb
\end{verbatim}
\item
  Install Node.js

\begin{verbatim}
sudo dpkg -i node_latest_armhf.deb
\end{verbatim}
\item
  Checkout source code

\begin{verbatim}
mkdir ~/src
cd src/
git clone <tresh_repo_url>
\end{verbatim}
\end{enumerate}

\subsection*{Configuration}\label{configuration}

\subsubsection*{DNS}\label{dns}

Add following line to network interface in /etc/network/interfaces:

\begin{verbatim}
dns-nameservers 8.8.8.8 8.8.4.4
\end{verbatim}

\subsubsection*{SSH Reverse Tunnel}\label{ssh-reverse-tunnel}

\begin{enumerate}
\def\labelenumi{\arabic{enumi}.}
\item
  Add autossh-tresh systemd service

\begin{verbatim}
pi@raspberrypi:~ $ vim /etc/systemd/system/autossh-tresh.service

[Unit]
Description=Autossh tresh: Maintain a ssh reverse tunnel to a control server
# After=network.target
After=network-online.target

[Service]
# User=autossh
User=pi
Group=pi
# -p [PORT]
# -l [user]
# -M 0 --> no monitoring
# -N Just open the connection and do nothing (not interactive)
# LOCALPORT:IP_ON_EXAMPLE_COM:PORT_ON_EXAMPLE_COM
# ExecStart=/usr/bin/autossh -M 0 -N -q \
# -o "ServerAliveInterval 60" -o "ServerAliveCountMax 3" \
# -p 22 -l autossh remote.example.com \
# -L 7474:127.0.0.1:7474 -i /home/autossh/.ssh/id_rsa
TimeoutStartSec=3
TimeoutStopSec=3
ExecStart=/usr/bin/autossh -M 20000 -N -C -q \
-o "ServerAliveInterval 60" -o "ServerAliveCountMax 3" \
-p 22 -l tooreht tjmp -R 22000:localhost:22 \
-i /home/pi/.ssh/id_rsa
# ExecStart=/home/pi/scripts/autossh-tresh.sh
Restart=always

[Install]
WantedBy=multi-user.target
\end{verbatim}
\item
  Enable service

\begin{verbatim}
systemctl enable autossh-tresh.service
\end{verbatim}
\end{enumerate}

\subsubsection*{Detect Waspmote Gateaway USB via
udev}\label{detect-waspmote-gateaway-usb-via-udev}

\begin{enumerate}
\def\labelenumi{\arabic{enumi}.}
\item
  Add udev rule

\begin{verbatim}
pi@raspberrypi:~ $ vim  /etc/udev/rules.d/99-tresh.rules

SUBSYSTEM=="tty", ACTION=="add", \
ATTRS{manufacturer}=="FTDI", \
ATTRS{product}=="FT232R USB UART",\
  PROGRAM="/bin/systemd-escape \
  -p --template=tresh-gateway@.service $env{DEVNAME} $env{ID_FS_LABEL}",\
  ENV{SYSTEMD_WANTS}+="%c"
\end{verbatim}
\item
  Add tresh-gateway@ systemd service

\begin{verbatim}
pi@raspberrypi:~ $ vim /etc/systemd/system/tresh-gateway@.service

[Unit]
Description=tresh siot gateway
BindTo=%i.device
After=%i.device
After=rc-local.service

[Service]
Type=oneshot
TimeoutStartSec=0
ExecStart=/home/pi/scripts/tresh-gateway.sh /%I
# RemainAfterExit=yes
\end{verbatim}
\item
  Prepeare adding bash script

\begin{verbatim}
mkdir -p /var/log/tresh
touch /var/log/tresh/tresh-gateway.log
\end{verbatim}
\item
  Add bash script tresh-gateway.sh

\begin{verbatim}
vim ~/scripts/tresh-gateway.sh

#!/bin/bash
LOG=/var/log/tresh/tresh-gateway.log
cd /home/pi/src/tresh/gateway/tresh-siot-gateway
DEVNAME=$1 node multi-sensor.js $1 | tee $LOG
\end{verbatim}
\item
  Set permissions

\begin{verbatim}
chmod  u+x /home/pi/src/tresh/gateway/tresh-siot-gateway
\end{verbatim}
\item
  Check USB insertion via systemd journal

\begin{verbatim}
journalctl -xn --follow
\end{verbatim}
\end{enumerate}

\subsubsection*{Connect to WPA
Enterprise}\label{connect-to-wpa-enterprise}

\begin{enumerate}
\def\labelenumi{\arabic{enumi}.}
\item
  Add wpa configs

\begin{verbatim}
vim /etc/wpa_supplicant/wpa_supplicant.conf
country=CH
ctrl_interface=DIR=/var/run/wpa_supplicant GROUP=netdev
update_config=1

# Locked BFH network
network={
    ssid="bfh"
    key_mgmt=WPA-EAP IEEE8021X
    eap=PEAP
    auth_alg=OPEN
    phase1="peaplabel=0"
    phase2="auth=MSCHAPV2"
    identity="insert-your-bfh-identity-here"
    password="insert-your-password-here"
    priority=100
}

# Open BFH network
network={
    ssid="public-bfh"
    key_mgmt=NONE
    priority=0
}
\end{verbatim}
\item
  Reboot

\begin{verbatim}
sudo reboot
\end{verbatim}
\item
  Check status

\begin{verbatim}
sudo wpa_cli status

Selected interface 'wlan0'
bssid=b8:be:bf:b7:e3:f0
freq=2412
ssid=bfh
id=0
mode=station
pairwise_cipher=CCMP
group_cipher=CCMP
key_mgmt=WPA2/IEEE 802.1X/EAP
wpa_state=COMPLETED
ip_address=147.87.46.112
p2p_device_address=b8:27:eb:aa:73:e7
address=b8:27:eb:aa:73:e7
Supplicant PAE state=AUTHENTICATED
suppPortStatus=Authorized
EAP state=SUCCESS
selectedMethod=25 (EAP-PEAP)
EAP TLS cipher=ECDHE-RSA-AES256-SHA
EAP-PEAPv0 Phase2 method=MSCHAPV2 uuid=edcd5979-ee8d-52ec-9c5e-5d3b6ed664c2
\end{verbatim}
\end{enumerate}

\subsection*{Misc}\label{misc}

\subsubsection*{Access RPI via ssh reverse
tunnel}\label{access-rpi-via-ssh-reverse-tunnel}

\begin{verbatim}
1. eval `ssh-agent -s`
2. ssh-add
3. ssh -N -R 22000:localhost:22 tjmp or autossh -M 20000 -f -N tjmp -R 22000:localhost:22 -C
ssh -p 22000 pi@localhost
\end{verbatim}