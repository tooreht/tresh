\chapter{Anwendungsfälle für \gls{iot}}\label{Anwendungsfälle für IoT}

\gls{iot} ist überall und doch (noch) nirgendwo. Damit \gls{iot} einen ökonomischen Nutzen bringt, werden Anwendungsfälle benötigt, welche den Anwendern einen effektiven Mehrwert bringen und den diese auch als solchen wahrnehmen. Gewisse \glqq{}Smartness\grqq{} ist bereits heutzutage vorhanden. Beispielsweise eine automatische Tür eines Einkaufsladens ist nichts anderes als ein Bewegungssensor, welcher als Trigger für einen Aktor (die automatische Tür) dient. Der Sensor ist in diesem Fall direkt beim Aktor. Sensor und Aktor sind nur untereinander verbunden, haben aber keine Verbindung zum \glqq{}Rest der Welt\grqq{}.
Nicht ohne Grund steht im Begriff \gls{iot} aber auch das \glqq{}I\grqq{} für Internet. Dank potenterer Hardware und neuer Kommunikationstechnologien ist es nun möglich, dass Sensoren und Aktoren nicht mehr direkt beieinander sein müssen. Stattdesswen werden Sie ans Internet angeschlossen und können dadurch weltweit vernetzt und kombiniert werden. Mit \gls{lora} ist neu eine energiesparende, kabellose Übertragungstechnik entwickelt worden, welche es ermöglicht Sensoren und Aktoren autonom über Monate oder gar Jahre zu betreiben, ohne dass diese direkt am Stromnetz angeschlossen sein müssen. Dank der Arbeit von Pascal Bohni und Roger Jaggi \autocite[29]{bfh:optimizedDataTransmission} wissen wir, dass mit \gls{lora} im Freien Übertragungen über mehrere Kilometer möglich sind und auch in Gebäuden rekordverdächtige Distanzen erreicht werden. Als nächstes werden grundsätzlich Use Cases über verschiedene Branchen hinweg analysiert.


\section{Smart Aggriculture}
Auf dem Bauernhof sind grosse Distanzen quasi an der Tagesordnung. Pflanzen gedeihen auf einem Acker. Auf einem anderen Stück Land weiden Kühe. Auf einem dritten Stück Land giesst der Bauer die Jauche aus. Diese Landstücke können gut mal ein Kilometer lang sein und auch über mehrere Kilometer verteilt sein. Mithilfe von \gls{lora} wäre es möglich, den Boden der Pflanzen auf Feuchtigkeit zu überwachen und bei Trockenheit ein automatisches Bewässerungssystem zu starten. Weiter können auch bestimmte Nährstoffe im Boden überwacht werden. Der Bauer würde informiert, sobald ein Nährstoff knapp wird, damit er diesen nach-düngen kann. Oder aber das System kann sogar selber die Düngstoffe verteilen. Beim Jauchenfass oder in Silos können Werte von Giftgasen überwacht werden und ein Alarm kann ausgelöst werden, wenn ein Gas eine zu hohe Konzentration erreicht. Mehrere Verteilte Wetterstationen können dem Bauern genaue Wetterangaben liefern, sowie Auswertungen erstellen, wie viel Sonne eine Pflanze erhielt und danach vergleichen wie gut sie gewachsen ist.

\section{Smart Industry}
Im Industriesektor wird \gls{iot} auch als die vierte industrielle Revolution bezeichnet. Die Produktionsstrassen sollen neu mithilfe von Sensoren überwacht werden. Aktoren reagieren auf Probleme oder mangelnde Qualität. Der klassische Produktionsmitarbeiter, der die Produktionsstrasse überwacht, entfällt dadurch. Weiter sollen die Maschinen menschlicher werden in dem die Motoren \glqq{}weicher\grqq{} werden. Das heisst, wenn ein Roboterarm beispielsweise bei seiner Bewegung auf einen Widerstand stösst, so soll er nicht mit voller Kraft versuchen diesen Widerstand zu überwinden, sondern stattdessen einen neuen Weg für die Bewegung suchen oder aber die Bewegung pausieren. Weiter können die Aktoren miteinander kommunizieren. Damit kann ein fehlerhafter Aktor beispielsweise sein Problem anderen Aktoren melden. Ein anderer Aktor kann die Funktion des defekten Aktors übernehmen bis dieser repariert wurde. Falls bei der Produktion giftige Flüssigkeiten oder Gase austreten, kann dies ebenfalls von Sensoren erfasst und entsprechend ein Alarm ausgelöst werden.

\section{Smart Logistics}
In der Logistik bietet \gls{iot} eine Menge Potential in der Automatisierung und Optimierung von Transporten. So können ganze Lager komplett von Robotern verwaltet werden, was neue Lagerungstechniken mit höheren Lagerungsdichte erlaubt. Dazu kommt eine schnellere Auslieferungszeit, da ein Lagerungssystem nicht zuerst nachfragen muss, wo jetzt etwas gelagert ist. Es hat seine Datenbank, welche darüber innerhalb weniger Millisekunden Auskunft geben kann. Auch bei der Auslieferung gibt es neue Möglichkeiten: Autonome (Flug-)Vehikel können einzelne Artikel individuell ausliefern, es müssen keine Kollektiv-Lieferungen gemacht werden. Beim Verkauf von Waren kann kontaktlos bezahlt werden und der Warenbestand ist immer aktuell.

\section{Smart Health}
In der Gesundheitsbranche gehören Sensoren grundsätzlich zur Tagesordnung. Puls-, EKG- und Sauerstoffüberwachungen an Patienten sind keine Neuigkeit mehr. Beispielsweise könnten aber auch die Schlafphasen der Patienten überwacht werden, um deren Schlafqualität zu analysieren. Weiter könnte zeitlich korrekte Medikamenteneinnahmen kontrolliert werden. Stürze von Patienten können detektiert und an das Personal oder Bekannte gemeldet werden, damit der Patient nicht Stunden- oder gar Tagelang am Boden liegen bleibt. Dazu könnte sich der Boden an der Sturzstelle angenehm aufwärmen, damit der umgefallene Patient nicht friert. Die Überwachung von Ultraviolett-Dosen, welche ein Patient erhält ist eine weitere Möglichkeit. Auch Temperaturen von medizinischen Kühlschränken können kontinuierlich und genau überwacht werden. Beim Ausfall eines Kühlaggregats könnte beispielsweise ein Notaggregat gestartet, oder der Inhalt des Kühlschranks in einen anderen Kühlschrank transportiert werden.

\section{Smart Education}
Der Lernprozess jedes Schülers kann individuell erfasst werden. Aufgrund dieser Daten erhält der Schüler auch immer entsprechende Aufgaben und Lernthemen, welche ihn ansprechen. Schüler können Projekte durchführen, von welchen der Stand jederzeit erfasst wird und von Lehrpersonen beaufsichtigt werden kann. Auch die Administration von den Stundenplänen kann in vielen Bereichen automatisiert oder optimiert werden. Zimmerreservationen können dynamisch verwaltet werden. Die Zimmer zeigen ihren aktuellen Stundenplan jeweils direkt beim Eingang auf einem Display an. Auch Austauschjahre können besser organisiert werden, da Schüler anhand ihrer Interessen einen guten Tauschpartner finden können.

\section{Smart Home}
Der Begriff \glqq{}Smart home\grqq{} ist einer der wohl bekanntesten im Zusammenhang mit \gls{iot}. In diesem Bereich haben sich schon verschiedenste Use Cases als Produkte etabliert. Der smarte Kühlschrank und die smarte LED-Lampe sind schon von verschiedensten Herstellern verfügbar. Grund dazu ist vermutlich, dass die Infrastruktur von Wohnungen sehr gute Grundvoraussetzungen für \gls{iot} zur Verfügung stellt. \gls{wlan} stellt eine Datenanbindung zur Verfügung und Strom kann aus jeder Steckdose bezogen werden. Trotzdem sind auch in diesem Bereich noch viele Anwendungsfälle möglich. Eine automatische Heizung welche die Wohnung dynamisch beheizt, abhängig von der An- oder Abwesenheit der Bewohner. Automatische Deaktivierung unnötiger Standby-Geräte wenn Bewohner schlafen. Wohnhausübergreifende Planung von Lärm-emittierenden Vorgängen wie z.B. Staubsaugen ist eine weitere Möglichkeit. Ein intelligenter Wecker könnte beispielsweise die zu weckende Person in einer optimalen Schlafphase mit Sonnenlicht wecken, indem er die Rollläden hochfährt. Bei schlechtem Wetter müsste er mit künstlichem Licht den Sonnenaufgang simulieren. Wenn dann die Person nach einer bestimmten Zeit noch nicht aufgestanden ist kann der Wecker beispielsweise auch noch Musik abspielen um die Person definitiv aus dem Schlaf zu holen.

\section{Smart City}
Städte können an beliebig vielen Stellen \glqq{}smart\grqq{} gemacht werden. Im Vergleich zum Smart Home gibt es in der Smart City nicht diese Schönwetter-Bedingungen für \gls{iot}. Strom ist nicht überall verfügbar und noch weniger ist es ein (kostenloser) Datenübertragungsanschluss. Deswegen gibt es aber nicht weniger mögliche Einsatzmöglichkeiten für \gls{iot}. Smarte Parkplätze, welche melden ob sie frei oder besetzt sind, sind beispielsweise nur der Anfang. Später könnte darauf sogar ein Reservationssystem für Parkplätze aufgebaut werden. Dazu müssen die Parkplätze sich selber natürlich mit einem Aktor blockieren können, damit ein reservierter Parkplatz nicht von einem anderem Auto besetzt wird. Audiosensoren können sogenannte Lärmkarten erstellen, welche wiederum zur Bewertung von Wohnlagen verwendet werden kann oder möglicherweise sogar von der Polizei für Risikozonen verwendet werden könnte. Aufgrund von Smartphone-Erkennung könnten Nadelöhre in Fussgängerzonen sowie Strassen erkannt werden und mit intelligenten Navigationssystemen könnten in Echtzeit optimale Routen für alle Fahrzeuge und Fussgänger berechnet werden.
\iffalse
  \subsection{Vision}
  Es ist 06:52 Uhr morgens. Die Sonne blinzelt über den Horizont und strahlt den Helligkeitssensor von Tom's Wohnblock an. Es ist Zeit zum aufstehen und Tom erreicht gerade eine ideale Schlafphase zum aufwachen. Deshalb werden soeben die Sonnenstoren von Tom's Schlafzimmer geöffnet um ihn mit Tageslicht zu wecken. Je nach Wetter würde zusätzlich noch das Licht des Schlafzimmers langsam hochgedimmt. Tom ist jedoch ein Siebenschläffer und bleibt trotzdem, dass er jetzt wach ist noch im Bett liegen. Da er 15 Minuten später noch mit keinem Fuss den Boden berührte wird nun Weckmusik gestartet. Diese Musik ist für Tom das Signal, dass er jetzt wohl oder übel sein Bett verlassen sollte.  Nach seinem morgendlichen Aufwachritual in Form von Duschen, Frisieren etc. hohlt er sich einen Becherkaffee aus dem Kühlschrank. Es ist der letzte Becher, doch Tom bemerkt das kaum noch, denn der Kühlschrank hat bereits Nachschub bestellt. Mit dem Kaffee verlässt er schliesslich seine Wohnung. Aufgrund seiner Abwesenheit werden nun automatisch die Storen geschlossen und alle nicht benötigten Systeme ausschaltet. Auch die Tür schliesst sich von selber. Leider musste er sein Auto am Vortag etwas weiter vom Block entfernt parkieren. Offenbar hatte jemand viel Besuch am Vorabend weshalb die Parkplätze direkt vor dem Block alle reserviert waren. Sein Fahrzeug hatte ihn bei der Fahrt nach Hause bereits über die reservierten Parkplätze informiert und ihn automatisch zum nächsten freien Parkplatz navigiert. Er könnte auch einen Aufpreis zahlen um einen Parkplatz vor dem Block fix für sich zu reservieren, dies ist ihm aber das Geld nicht wert. Ausserdem soll ja sowieso bald endlich der komplette Autonome Modus seines Tesla verfügbar sein. Dieser fährt dann direkt vor die Haustür sobald Tom seine Wohnung verlässt. Unterwegs zum Auto wirft er den mittlerweile leergetrunkenen Becher in einen Mülleimer. Der Besuch, der die Parkplätze besetzte, hatte anscheinend auch draussen üppig gefeiert, denn der Mülleimer ist mit seinem Kaffeebecher nun definitiv randvoll. Er steigt in seinen Tesla ein welcher im sogleich mitteilt, dass er über Nacht komplett vollgeladen wurde vom im Parkfeld integrierten Induktionsfeld. Die Kosten für den Parkplatz und den Ladestrom welche der Tesla über Nacht gebraucht hat, werden Tom direkt seinem Konto berechnet. Tom ist das recht, so hat er sich um etwas weniger zu kümmern. Zufrieden fährt er mit dem Tesla davon, wissend, dass dieser bereits die optimalste Route zu seinem Arbeitsplatz geplant hat. Kaum hat er seinen Parkplatz verlassen parkiert auf diesem ein Müllabfuhr-Fahrzeug. Ein Müllmann steigt aus und geht sogleich zu dem vollen Mülleimer um ihn zu entleeren. Eigentlich ist dieser Mülleimer sehr unpassend für seine Mülltour, da dieser aber schon voll ist hat das System einen Umweg eingeplant, um diesen \glqq{}Notfall\grqq{} schnellstmöglich zu lösen. Dem Müllmann ist das egal, denn seit dem neuen Mülltouren-Planer fährt er nicht mehr sinnlos zu leeren Mülleimern und kann diese Zeit stattdessen mit spannenderen Arbeiten verbringen.
\fi