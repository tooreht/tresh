\chapter{Fazit}

Wir haben im Rahmen dieser Projekt 2 Arbeit viele Erfahrungen zum Thema \gls{iot} gesammelt. Sei es die Auseinandersetzung mit möglichen Use-Cases, der Definition von Rahmenbedingungen für \gls{iot}, dem grundlegenden Aufbau einer \gls{iot}-Architektur, dem Herausfiltern der wichtigsten Anforderungen an \glspl{iotk}, ein Überblick über die verfügbaren \gls{iot} Funktechnologien und deren Vor- und Nachteile und schliesslich der Implementation und Evaluation einer eigenen kleinen \gls{iot}-Infrastruktur mit \gls{lora} anhand des Tresh Use-Case. Durch all diese Bereiche konnten wir \gls{iot} also aus verschiedenen Blickwinkeln betrachten und ein Gefühl für das Potential, die Herausforderungen und die Limitationen von \gls{iot} entwickeln.\\
Sehr interessant und lehrreich war natürlich auch die Implementation, da in diesem Moment die physikalischen Grenzen plötzlich zum Vorschein kommen, welche in der Theorie selten beachtet werden. In unserem Fall war das beispielsweise der Metall-PET-Eimer, welchen wir aufgrund seiner starken Dämpfung nicht mit dem Sensorknoten ausrüsten konnten. Die Lösung - ein Koaxialkabel für die Antenne - hätten wir zwar gefunden, leider aber keine Zeit mehr für deren Installation.\\
Im theoretischen Teil bereuen wir ein wenig, dass wir erst gegen Ende der Arbeit festgestellt haben, dass wir \gls{lorawan} zu wenig vertieft angeschaut haben. Stattdessen wollten wir die beiden verschieden Ansätze von \gls{lora} und \gls{zigbee} im Praxistest untersuchen. Aus zeitlichen Gründen haben wir dann ausschliesslich mit \gls{lora} gearbeitet. Wir entschieden uns für \gls{lora}, da die Mesh-Topologie im Zusammenhang mit Low-Energy-Geräten nicht sinnvoll ist.
Alles in allem haben  wir aber viel gelernt in der Arbeit, gerade im Bereich \gls{lora} und sind somit mit unser Leistung zufrieden. Wir denken, dass wir mit dieser Arbeit eine gute Grundlage geschaffen haben, um die ganze Thematik im Rahmen unserer Bachelor-Thesis zu vertiefen.
