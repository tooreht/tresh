\newglossaryentry{BibTeX}{%
	name={BibTeX},%
    description={Programm zur Erstellung von Literaturangaben und -verzeichnissen in \TeX- oder \LaTeX-Dokumenten}}

\newglossaryentry{iotk}{%
	name={IoT-Knoten},%
    plural={IoT-Knoten},
    description={Komponente bestehend aus einem Microcontroller der  Sensoren und Aktoren steuert und die Daten über Schnittstellen zur Verfügung stellt}}
    
\newglossaryentry{iotg}{%
	name={IoT-Gateway},%
    plural={IoT-Gateway},
    description={Komponente bestehend aus Funk-Technik für die Kommunikation mit \gls{iotk} und Netzwerkschnittstelle für die Kommunikation mit \gls{iotp}}}
    
\newglossaryentry{iotp}{%
	name={IoT-Plattform},%
    plural={IoT-Plattform},
    description={Cloud-Produkt welches die Daten von \gls{iotk} entgegennimmt, aggregiert und dem Benutzer präsentiert oder damit Aktoren steuert.}}
    
\newglossaryentry{ism}{
	name={ISM-Band},
    plural={ISM-Bänder},
    description={Als ISM-Bänder (Industrial, Scientific and Medical Band) werden Frequenzbereiche bezeichnet, die durch Hochfrequenz-Geräte in Industrie, Wissenschaft, Medizin, in häuslichen und ähnlichen Bereichen genutzt werden können. Entsprechende ISM-Geräte wie Mikrowellenherde und medizinische Geräte zur Kurzwellenbestrahlung benötigen dabei nur eine allgemeine Zulassung der Frequenzverwaltung, beispielsweise in Deutschland.}}
    
\newglossaryentry{lora}{
	name={LoRa},
    plural={LoRa},
    description={LoRa (LongRange) ist eine proprietäre CSS modulations technologie verwendet für LPWAN/LoRaWAN patentiert von Semtech von der LoRa Allianz}}

\newglossaryentry{lorawan}{
	name={LoRaWAN},
    plural={LoRaWANs},
    description={LoRaWAN (Long Range Wide Area Network) ist ein Low-Power-Wireless-Netzwerkprotokoll, das für die sichere bidirektionale Kommunikation im Internet der Dinge (IoT) entwickelt wurde. LoRaWAN basiert auf dem offenen Industrie-Standard LoRa und wird von der non-profit Organisation LoRa Alliance spezifiziert.}}
    
\newglossaryentry{pshmthd}{
	name={Push-Methode},
    plural={Push-Methoden},
    description={Die Push-Methode ist eine betriebswirtschaftlicher Begriff für eine Marketing-Strategie. Die Push-Methode entwickelt oder evaluiert zuerst eine neue Technologie. Mithilfe dieser Technologie wird danach nach Bedürfnissen gesucht, wo diese Technologie wirtschaftlich eingesetzt werden kann.}}

\newglossaryentry{wifi}{
	name={WiFi},
    plural={LoRaWANs},
    description={LoRaWAN (Long Range Wide Area Network) ist ein Low-Power-Wireless-Netzwerkprotokoll, das für die sichere bidirektionale Kommunikation im Internet der Dinge (IoT) entwickelt wurde. LoRaWAN basiert auf dem offenen Industrie-Standard LoRa und wird von der non-profit Organisation LoRa Alliance spezifiziert.}}

\newglossaryentry{bluetooth}{
	name={Bluetooth},
    plural={LoRaWANs},
    description={LoRaWAN (Long Range Wide Area Network) ist ein Low-Power-Wireless-Netzwerkprotokoll, das für die sichere bidirektionale Kommunikation im Internet der Dinge (IoT) entwickelt wurde. LoRaWAN basiert auf dem offenen Industrie-Standard LoRa und wird von der non-profit Organisation LoRa Alliance spezifiziert.}}


\newacronym{iot}{IoT}{Internet of Things}
\newacronym{ota}{OTA}{Over-the-Air}


