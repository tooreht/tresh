\newglossaryentry{BibTeX}{%
	name={BibTeX},%
    description={Programm zur Erstellung von Literaturangaben und -verzeichnissen in \TeX- oder \LaTeX-Dokumenten}}

\newglossaryentry{iotk}{%
	name={IoT-Knoten},%
    description={Komponente bestehend aus einem Microcontroller der  Sensoren und Aktoren steuert und die Daten über Schnittstellen zur Vefügung stellt}}
    
\newglossaryentry{ism}{
	name={ISM-Band},
    plural={ISM-Bänder},
    description={Als ISM-Bänder (Industrial, Scientific and Medical Band) werden Frequenzbereiche bezeichnet, die durch Hochfrequenz-Geräte in Industrie, Wissenschaft, Medizin, in häuslichen und ähnlichen Bereichen genutzt werden können. Entsprechende ISM-Geräte wie Mikrowellenherde und medizinische Geräte zur Kurzwellenbestrahlung benötigen dabei nur eine allgemeine Zulassung der Frequenzverwaltung, beispielsweise in Deutschland.}}
    
\newglossaryentry{lora}{
	name={LoRa},
    plural={LoRa},
    description={LoRa (LongRange) ist eine proprietäre CSS modulations technologie verwendet für LPWAN/LoRaWAN patentiert von Semtech von der LoRa Allianz}}

\newglossaryentry{lorawan}{
	name={LoRaWAN},
    plural={LoRaWANs},
    description={LoRaWAN (Long Range Wide Area Network) ist ein Low-Power-Wireless-Netzwerkprotokoll, das für die sichere bidirektionale Kommunikation im Internet der Dinge (IoT) entwickelt wurde. LoRaWAN basiert auf dem offenen Industrie-Standard LoRa und wird von der non-profit Organisation LoRa Alliance spezifiziert.}}

\newglossaryentry{wifi}{
	name={WiFi},
    plural={LoRaWANs},
    description={LoRaWAN (Long Range Wide Area Network) ist ein Low-Power-Wireless-Netzwerkprotokoll, das für die sichere bidirektionale Kommunikation im Internet der Dinge (IoT) entwickelt wurde. LoRaWAN basiert auf dem offenen Industrie-Standard LoRa und wird von der non-profit Organisation LoRa Alliance spezifiziert.}}

\newglossaryentry{bluetooth}{
	name={Bluetooth},
    plural={LoRaWANs},
    description={LoRaWAN (Long Range Wide Area Network) ist ein Low-Power-Wireless-Netzwerkprotokoll, das für die sichere bidirektionale Kommunikation im Internet der Dinge (IoT) entwickelt wurde. LoRaWAN basiert auf dem offenen Industrie-Standard LoRa und wird von der non-profit Organisation LoRa Alliance spezifiziert.}}


\newacronym{iot}{IoT}{Internet of Things}
\newacronym{ota}{OTA}{Over-the-Air}


